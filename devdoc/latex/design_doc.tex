\documentclass[a4paper,10pt]{book}
\usepackage[T1]{fontenc}
\usepackage[utf8]{inputenc}
\usepackage{mathtools}
\usepackage{minted}
\usepackage{hyperref}
\usepackage[table]{xcolor}
\usepackage{tikz}

\definecolor{red}{HTML}{FF0000}
\definecolor{globaltablecolor}{HTML}{FF0000}
\definecolor{blue}{HTML}{0000FF}
\definecolor{orange}{HTML}{FF8000}

\begin{document}
% new commands -----------------------------------------------------------------
% - name variables -------------------------------------------------------------
\newcommand{\Projectname}{Project Crucible}
\newcommand{\Enginename}{Crucible}
\newcommand{\Gamename}{Library of Worlds}
\newcommand{\enginenamespace}{cru::}
% - admonitions ----------------------------------------------------------------
%#1 - title and border color
%#2 - title text color
%#3 - background color
%#4 - text color
%#5 - title text
%#6 - contents text
%#7 - symbol to be displaayed on left border
\newcommand{\admonition}[7]{
    \tikzstyle{thebox} = [draw=#1, fill=#3, text=#4, very thick,
    rectangle, rounded corners, inner sep=10pt, inner ysep=20pt]
    \tikzstyle{thetitle} =[fill=#1, text=#2]

    \begin{tikzpicture}
        \node [thebox] (box){
            \begin{minipage}{0.9\textwidth-5\marginparsep}
                #6
            \end{minipage}
        };
        \node[thetitle, rounded corners, right=10pt] at (box.north west) {#5};
        \node[thetitle, rounded corners] at (box.west) {#7};
    \end{tikzpicture}
}

\newcommand{\TODO}[1]{
    \admonition{red}{white}{red!10}{red}{TODO}{#1}{$!$}
}
\newcommand{\danger}[1]{
    \admonition{red}{white}{red!10}{red}{DANGER}{#1}{$!$}
}
\newcommand{\note}[1]{
    \admonition{blue!60}{black}{blue!10}{black}{NOTE}{#1}{$?$}
}
\newcommand{\think}[1]{
    \admonition{orange}{white}{orange!15}{orange}{THINK}{#1}{$?$}
}

% - grids ----------------------------------------------------------------------
%isometric grid
%#1 - tiles per side
\newcommand{\isogrid}[1] {
\begin{tikzpicture}
  \foreach \i in {0, ..., #1} {
    \draw (#1/2 + \i/2, #1 - \i/4) node[anchor=south west] {$\i$}
       -- (0 + \i/2, #1/1.33 - \i/4) node[anchor=north east] {$\i$};
    \draw (#1/2 - \i/2, #1 - \i/4) node[anchor=south east] {$\i$}
       -- (#1 - \i/2, #1/1.33 - \i/4) node[anchor=north west] {$\i$};
  }
\end{tikzpicture}
}

\newcommand{\isogridsub}[1] {
\begin{tikzpicture}
  \foreach \i in {0, 2, ..., #1} {
    \draw (#1/2 + \i/2, #1 - \i/4) node[anchor=south west] {$\i$}
       -- (0 + \i/2, #1/1.33 - \i/4) node[anchor=north east] {$\i$};
    \draw (#1/2 - \i/2, #1 - \i/4) node[anchor=south east] {$\i$}
       -- (#1 - \i/2, #1/1.33 - \i/4) node[anchor=north west] {$\i$};
  }
  \foreach \i in {1, 3, ..., #1} {
    \draw[gray,dashed] (#1/2 + \i/2, #1 - \i/4) node[anchor=south west] {$\i$}
       -- (0 + \i/2, #1/1.33 - \i/4) node[anchor=north east] {$\i$};
    \draw[gray,dashed] (#1/2 - \i/2, #1 - \i/4) node[anchor=south east] {$\i$}
       -- (#1 - \i/2, #1/1.33 - \i/4) node[anchor=north west] {$\i$};
  }
\end{tikzpicture}
}
% - other ----------------------------------------------------------------------
\newcommand{\textbi}[1]{\textbf{\textit{#1}}}
%codeword: mboxing to prevent hyphenation, macro is meant for short code commands only
\newcommand{\codew}[1]{\mbox{\texttt{#1}}}

\author{Giles Egidijus Nedzinskas}
\title{Design Document for \Projectname{}}
\maketitle{}

\tableofcontents

\part{Features}

\chapter{Overview}
\Projectname{} is a project encompassing a game-engine (\Enginename{}) and a
game written on it (\Gamename{}).

\Projectname{} begins with the two-pronged passion of one person that for one -
thirsts for knowledge and technical prowess in programming and believes that a
game-engine is a great and multi-faceted enough challenge to embark on in hopes
of becoming a better professional in the software development industry, and two
- has a great passion for games and fiction and hopes to use the game as a
creative outlet in addition to the creative challenge provided by game engine
development.

\Gamename{} is a tile-based, turn-based RPG, inspired by the looks of games
made in the 90s and the look and feel of many RPG games made since then. The
game is developed alongside the engine \Enginename{} and together with it forms
the main and biggest portion of \Projectname{}.

Work on \Projectname{} started at the very beginning of the year 2019 in the
form of scattered research and various technical experiments. The first lines of
this document were written in the morning of 2019-04-05.

\section{Planned Features}
\subsection{General Gameplay}
\begin{itemize}
  \item Tiles
  \item Turns
  \item Party based (the player can -- and is encouraged to -- control more
    than one character at once)
  \item Passage of time
  \item Time of day and the fun associated with it (e.g. you can hide better at
    night, but can allso be caught unawares if not careful)
  \item Long voyages via different means of transportation
  \item Basic crafting (enchanting, enhancing player gear, etc)
  \item Lighting levels (e.g. tiles closer to a torch are better lit, caves
    are naturally dark, etc)
  \item Walls between tiles - the walls do not have to occupy an entire tile
\end{itemize}

\subsection{Combat}
\begin{itemize}
  \item magic
  \item area of effect spells/weapons (long duration, instant, hidden(traps -
    could be a trap-spell like a firemine-rune or something more mundane))
  \item long range, mele, thrown weapons
  \item squad tactics
  \item basic cover system (e.g. less accuracy if projectile crosses a non-flat
    item)
\end{itemize}

\subsection{Graphics}
\begin{itemize}
  \item isometric tiles
  \item SDL2 (possibly OpenGL later)
  \item Basic tile/item animations (e.g. torches, water tiles)
\end{itemize}

\subsection{Other}
\begin{itemize}
  \item Multi-genre gameplay: the game is not confined to a single setting and
    can span sci-fi and fantasy realms in the same grand campaign.
\end{itemize}

\section{Possible Features}
\begin{itemize}
  \item Isometric view
  \item Animated characters (e.g. movement, combat)
  \item Implementation in OpenGL. Currently plain SDL2 seems like a good
    choice, but I do want to learn OpenGL and it is quite more popular in the
    industry.
\end{itemize}

\section{Features that Will Not be Included}
\begin{itemize}
  \item \textbf{Real-time movement and/or combat.} Real-time is outside the
    scope of the project as it falls outside the feel that \Gamename{} aims to
    provide.
 
  Nevertheless the challenge of supporting pusable realtime sounds like an
    interesting and worth-while challenge and would be something quite
    interesting to implement in the future, maybe in \Enginename{} II.
 
  \item \textbf{3D.} The engine aims to recreate the look and feel of an era
    when 2D and pre-rendered 3D dominated.
\end{itemize}

\chapter{Workplan}
The project will most likely take us through a few iterations of prototypes
before we acquire enough knowledge and know-how to implement the final
framework of \Enginename{}.

\begin{itemize}
  \item \Enginename{} 0.1 (a.k.a Prototype I) should provide us with the
    possibility to create a few basic inter-connected levels. Thus the
    functionality we should have:
  \begin{itemize}
    \item a basic level editor
    \item a few tiles with simple graphics
    \item it should be possible to load assets from the filesystem (e.g.
      textures)
    \item definition of a starting point and a player-controlled character that
      can walk around
    \item we have to be able to save the level into a file (tile settings and
      textures, level info, etc)
  \end{itemize}
  \item \Enginename{} 0.2 (a.k.a. Prototype II) should allow us to implement
    player interaction with the gameworld
  \begin{itemize}
    \item doors, windows
    \item containers
    \item a way to program items into the game(weapons, armors) (this does not
      touch on in-game crafting)
    \item at least partially implemented combat (we can ommit damage types and
      similar things for now)
  \end{itemize}
\end{itemize}

\chapter{Combat}

\section{Mechanics}
Current (very early) idea of combat. Damage calculated for each damage type
separately using the following logic and equations.

\subsubsection*{Step 1}
Attack and defence scores are calculated (random value up to stat maximum):
\begin{equation*} atk \sim U([0,1 \dots atk_{max}) \end{equation*}
\begin{equation*} def \sim U([0,1 \dots def_{max}) + dodge\end{equation*}
The attack connects \begin{math} if(atk > def) \end{math}. 
\subsubsection*{Step 2}
If the attack is successfull , we calculate actual damage received. First,
incoming damage(based on weapon dmg dice) and armor values are rolled for the
current attack.
\begin{equation*}
  dmg_i =
  \Big(
    \sum\limits_{n=1}^{dicecnt}\big(diceroll(n)\big)
  \Big) + dmg_{const}
\end{equation*}
\begin{equation*} arm \sim U([0,1 \dots arm_{max}) \end{equation*}
Then armor is substracted from damage (including resistences or weaknesses
incurred by the armor (if worn)), and we have our final incoming damage to the
entity.  The final damage received is still subject to further modification due
to entity resistences or weaknesses to the incoming damage type.
\begin{equation*}
  dmg = (dmg_i - arm \times resist_{arm}) \div resist_{entity}
\end{equation*}

\subparagraph*{Notes}
\begin{itemize}
  \item This logic does not represent catering for damage types, I belive we can
    calculate each damage type separately calling the above logic once per
    damage type.
  \item the armor score is randomized to emulate lucky hits, etc. If we decide
    to suport hit-locations we may want to make the armor score non-random or
    reduce the range coloser to the max value (to still emulate some luck, then
    again luck can be represented with critical strikes).
\end{itemize}

\section{Item Examples}
\begin{table*}[h]
  \rowcolors{3}{globaltablecolor!6}{globaltablecolor!12}
  \begin{tabular}{l c c c l}
    %\firsthline
    \textbf{weapons} \\
    \textbi{name}      & \textbi{damage}     & \textbi{cost}     & \textbi{weight} & \textbi{notes} \\
    axe                & sw+2 cut   & \${}50   & 4      &       \\
    auto pistol 9mm    & 2d6+2 pi   & cost     & weight & [1]   \\
    %\lasthline
  \end{tabular}

  \rowcolors{3}{globaltablecolor!6}{globaltablecolor!12}
  \begin{tabular}{l c c c c l}
    %\firsthline
    \textbf{armor} \\
    \textbi{name}      & \textbi{locationq} & \textbi{DR}   & \textbi{cost} & \textbi{weight} & \textbi{notes} \\
    scale armor        & torso, groin & 4        & \${}420  & 35     &         \\
    ballistic suit     & body, limbs  & 12/$4^*$ & cost     & weight & [1,2,3] \\
    %\lasthline
  \end{tabular}
\end{table*}

\TODO{
  \begin{itemize}
    \item Make example combat situations with some armor and mele and ranged
      wepons
    \item Move example item tables to basic equipment section, add a few more
      items
    \item Make explanations of damage and DR notation
  \end{itemize}
}

\chapter{Taking Turns}
\TODO{
  Things to ponder
  \begin{itemize}
    \item How will we represent speed, ecpecially in combat (time units in
      combat mode?)
    \item we should probably have a peacefull mode and a combat mode (as we
      want to be able to move more quiclky while not in combat (go to mouse
      click), etc)
    \item what actions will take TUs (e.g. easy and hard to access inventory
      items, attacks, spell preparation, arming grenades, etc)
  \end{itemize}
}

\section{Combat}
Combat will be squad-style: the player team takes a turn, then the environment
takes a turn. Possibly we will want to support several teams, if -- for example
-- we have several hostile parties with different friend-foe logic and agenda,
like where there are guards that are hostile towards the player's company for
crime, but there are also bandits on the scene that are hostile towards both the
guards and the player's party, and there is allso that sneaking pack of starved
lions who want dinner and nobody noticed them yet.

\section{Peacefull Mode}
\begin{itemize}
\item resting
  \begin{itemize}
  \item fast movement (skip turns untill selected character reaches destination
    (mouseclick)
    \item no need to move each party member, they group around selected party
      member
    \item do we need formations? probably just gonna use line formation for
      simplicity.  If I recall correctly D\&D used to make people choose who is
      where in the line without ever mentioning oher formations. Even simpler,
      maybe party members should just try to be as close to the player as
      possible.
  \end{itemize}
\end{itemize}

\chapter{Game Levels}
\section{Static Pieces}
(e.g.  tiles, walls)
\section{Interactable Pieces}
(e.g. doors, windows, workbenches, drawbridges)
\section{Items, Containers, Furniture}
\section{Decorations}
(e.g. floor decals, etc)
\section{Destructability}
(e.g. burning bridge)
\section{Lighting Levels}
\TODO{This section will be about lighting levels and how it affects gameplay,
the associated abilities and some possible situations and circumstances of
gameplay interest}

\chapter{Travelling}
 \think{Maybe we should simplify travelling a bit? It will not happen all that
   much in the game, at least the vehicles part\dots{} will it?}
\section{Travel Speed, Transportation and Supplies}
\subsection{Seafairing}
\subsection{Passing Mountains}
\subsection{Travelling Through Forests}
\subsection{Deserts}
\subsection{Flying}
\subsection{Transport animals}
\subsection{Transport vehicles}
\subsubsection{Muscle Powered (e.g. bicycle or animal-drawn)}
\subsubsection{Powered}
Energy cells, solar, other natural source, fuel (e.g. gasolene). There are
different consideration for energy powered vehicles than fueled vehicles.
\subsection{Rest}
\subsection{Pathfinding}
\subsection{Tracking and Being Tracked}
\subsection{Time of Day}
\subsection{Interdimentional Travel}
\section{Difficult Terrain}
\subsection{Mountaineering}
\subsection{Forestry}
\subsection{Swamps}
\subsection{Desert Survival}
\section{Encounters}
\subsection{Area Danger and Peace Level}
\note{there may be a restless area where encounters are likely, but are less
  dangerous for a seasoned advenurer, but there may be peacefull places where
  encounters are rare but may be very dangerous when they do happen}
\subsection{Random Encounters}
\subsection{Preset Encounters}

\chapter{Other Features}

\section{Crafting, Enhancing and Enchanting}
Crafting will not be a large feature in the game, so we group it together with
other topics that change equipment properties.
\note{Maybe crafting will not be in the game at all.}

\part{The Engine}
\note{Often it is quite difficult to determine which components belong to the
engine and which to the game when the two are developed side-by-side, thus this
part should be reviewed and reevaluated periodically.}
\chapter{Overview}
The engine is implemented as a modular set of objects and functions accessible
under the
\codew{\enginenamespace{}} namespace, forming a framework for game development,
oriented at 2d isometric tile based RPG games and is targeted at the C++
programming language.
\chapter{Graphics}
We will be using SDL2 as our main graphics library for now, as we already have
a working codebase and some knowledge of working with the lib. I wish to
implement the graphics module of the engine in OpenGL later (not because we
necissarily needed it, but I do want to learn it some and I like the prospect
of a possibility to use some of the features in the future.).
\section{Character Movement}
There are two options currently being considered: \\
The map will consist of a grid of nodes, one node every half a tile. Thus,
wall segments can be placed on the nodes and every time the player moves to
another tile, he has to cross the node and we can check if the node is passable,
is there a trap there, etc. We can allso draw trigger regions, etc using this
node system.
\subparagraph{pros}
\begin{itemize}
  \item easy to calculate collissions by simply checking what is at the node
    being traversed.
  \item walls and a lot of various map items can be placed on and between tiles
    using this method (items, rectagular and even polygonal areas, etc).
  \item characters, items, triggers can be placed in door entrances, wall
    alcoves, etc. (e.g. wall traps, objects/bodies blocking doors from closing,
    etc)
\end{itemize}
\subparagraph{cons}
\begin{itemize}
  \item the tile coordinates will be a bit weird. Either $\displaystyle
    tile_{xy} = \frac{node_{xy}}{2}$ or $\displaystyle tile_{xy} =
    \frac{node_{xy}}{2} + 1$ depending on if we start the coordinate system
    at 1 or 0 respectively.
  \item this is not too bad, but the grid has a noticeable memory footprint
    e.g. 256*256 grid would weigh in at exactly 512kb per layer. And we will
    have multiple layers, unless we decide to point to arrays of map-obects
    from the grid nodes to avoid having multiple layers, this in turn may
    introduce unnecessary code complexity.
\end{itemize}
\note{so far the node-grid approach wins, having considered the following
approaches: walls-as-tiles, walls as 2d lines represnted by 4 32bit integrals}
\begin{figure}
  \caption{Gameworld grid (note the coordinates being at intersections
    \textbf{and} mid-tile)}
  \isogridsub{10}
\end{figure}
\section{Function \enginenamespace{}gfxinit}
This function should initialize the graphics module e.g. init the graphical lib,
create main window and content, etc.
\section{\enginenamespace{}Viewport}
The viewport is a way to represent a level map via isometric tiles and other
sprites.  The \enginenamespace{}Level is used to load, contain and update level
data, the \enginenamespace{}Viewport acts as a window to the Level and provides
a way to view the data stored in the \enginenamespace{}Level with options on
how to display said data (e.g. centner on certain character/object/etc,
show/hide/lower walls, draw frames around obscured character sprites (or arrows
above them, display/hide debug regions, etc).

The viewport accepts on-screen x/y position, height and width parameters to
display the tilemap on the screen and also accepts tile-position in order to
display the desired portion of the map.
\begin{table}[h]
  \rowcolors{2}{globaltablecolor!6}{globaltablecolor!12}
  \caption{Some Values of \enginenamespace{}Tilemap}
  \begin{tabular}{p{0.4\linewidth} p{0.6\linewidth - 1cm}}
    \textbf{Type} & \textbf{Note} \\
    \enginenamespace{}Rect m\_screen\_pos & where to draw the Tilemap on screen (in pixels) \\
    \enginenamespace{}Point m\_viewport\_pos & what part of the map to show (upper-left corner,
                                               tile coordinates) \\
    std::shared\_ptr<\enginenamespace{}Level> m\_level & pointer to the level data we will display \\
  \end{tabular}
\end{table}
\section{\enginenamespace{}Button}
Settings: on-screen position and size (Rect), state graphics (idle, clicked),
text, keyboard shortcut button (possibly optionally), an on-click function.
\section{\enginenamespace{}Checkbox}
Takes active and inactive state graphics.
\section{\enginenamespace{}Textbox}
Used to display rich text (colors, boldness, italics). \\
Supports word-wrapping, vertical scrolling. \\
Settings: text buffer to display, box borders - optionally, background -
optionally. I imagine the text format syntax to be quite simple, e.g. scoped
escape sequences a bit similar to LaTeX,
\begin{verbatim}
default text \color(red){\bold{ some red bold text \italic{my
red bold italic text} bold and red again} just red} default
text again
\end{verbatim}
like so.
The object keeps the text and the resulting texture that will be used by the
renderer.
\paragraph{possible text format parsing logic}
\begin{enumerate}
  \item Keep a render-flag-enable history (first-in-last-out).
  \item Found escape sequence = enable associated text rendering flag and add to
    enable history.
  \item Found end of scope = disable last enabled rendering flag and remove from
    history.
\end{enumerate}
\TODO{Document the text format syntax}
\chapter{Misc}
\TODO{Most things under this category should get sorted to other places, with time}
\section{\enginenamespace{}Program\_environment}
Stores a lot of variables needed throughout the program like program arguments,
gfx settings and possbly main assets, main window height and width,
text colors, rendering environment (library and possibly hardware associated),
custom cursor, etc.
\think{Should probably be divided into system-environment and game-environment}
\section{\enginenamespace{}Rect}
A rectangle: \texttt{int x, y, w, h}.
\section{\enginenamespace{}Point}
A two-dimentional point: \texttt{int x, y}.
\section{\enginenamespace{}Level}
Stores data about the level.
\subsubsection*{Among the members:}
\begin{itemize}
  \item m\_tiles - to store floor tiles (e.g. floors, water, lava, etc),
    basically the floor of the level (or lack thereof i.e. tile\_empty)
  \item m\_items - items (e.g. loot, containers, etc). Drawn on the tiles and decor.
  \item m\_walls - the walls. Drawn on top of everything on the tiles, can
    obscure things, maybe we should implement a cutout.
  \item m\_characters - everybody (people, animals, automatons) that has any
    kind of AI element. The player character(s) will most likely be in this array
    as well if we do not find a good reason not to put them here.
\end{itemize}
\section{\enginenamespace{}Character}
Char stats, AI type (none = player control), sprite, etc.
\TODO{}
\chapter{Combat}
\section{Function \enginenamespace{}diceroll}
The damage value should be passed to the damage calculation module in the form
of a dice notation e.g. \codew{3d6+2}, \codew{4d4+5}.
\note{currently the number of dice is not optional, even if only one dice is
  used. E.g. use \codew{1d6} insted of just \codew{d6}}
The number and sides of dice are not restricted e.g. \codew{1d27} is perfectly legal,
though it would be difficult to find such dice in reality.
\begin{minted}{c++}
  //pseudocode
  void cru::dice_to_int(const std::string& str)
  {
    try {
      size_t dpos = str.find("d");
      size_t pluspos = str.find("+");
      const int dice_cnt {std::stoi(str.substr(0, dpos-1))};
      const int dice_sides {std::stoi(str.substr(dpos+1, pluspos-1))};
      int add_val {0};
      if(pluspos != std::string::endl) {
        add_val = std::stoi(str.substr(pluspos+1));
      }
    } catch {
      errlog("dice_to_int: bad format\n");
    }
  }
\end{minted}
\danger{
  should make this part as simple as plausible.
  \begin{itemize}
    \item We may want to support something like 3d8+d6+5d15, but that is
      unnecessarily complicated for most use cases. If someone wants to go crazy
      with such values, let them do the work in the their game code.
    \item Support for dice values not beginning on 1 is allso likely not necessary.
  \end{itemize}
}
\note{
  dice notation format format: \\
  \codew{<n of dice>d<max dice value>[+<add value>]}
   \begin{tabular*}{0.9\linewidth}{rl}
     \textbf{example values:} \\
     good values: & \codew{1d6}, \codew{3d4}, \codew{6d5+3} \\
     bad values: & \codew{d6}, \codew{d4+3}, \codew{2d6+1d4}, \codew{3d6+d4} \\
   \end{tabular*}
}

\part{The Setting}
\TODO{gather the scattered notes here and start from there}
%could be more than one world, probably a chapter for each major world then
\chapter{World}
\section{Magic}
\section{Races}
\section{Items}
\subsection{Weapons}
\subsection{Armor}
\subsection{Arcane}
\chapter{Minor Races}
\chapter{Beasts}
\TODO{The chapter name ``Beasts'', implies less than we want to cover here,
  ``Bestiary'' or ``Other Creatures'' or ``Book of Creatures'' would be
  something more like it}
\section{The Wilds}
\section{The Underworld}
\section{The Void}

\part{The Story}
\TODO{gather the scattered notes here and start from there}

\end{document}
