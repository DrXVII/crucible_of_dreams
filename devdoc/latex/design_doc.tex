\documentclass[a4paper,10pt]{book}
\usepackage[utf8]{inputenc}
\usepackage{mathtools}
\usepackage{minted}
\usepackage{hyperref}
\usepackage{tabu}
\usepackage[table]{xcolor}

\definecolor{milkpink}{HTML}{FFEEEE}
\definecolor{rose}{HTML}{FFDDDD}
\definecolor{red}{HTML}{FF0000}

\begin{document}
% new commands ------------------------------------------------------------------
% - name variables --------------------------------------------------------------
\newcommand{\Projectname}{Project Crucible}
\newcommand{\Enginename}{Crucible}
\newcommand{\Gamename}{Library of Worlds}
% - special formats -------------------------------------------------------------
%TODO - find a way to left-allign the TODO section, explore other tex/latex
% boxes/containers, make similar functions for, blocks meant for reading,
% like NOTE, WARNING, INFO, TIP, SUGGESTION etc.
\newcommand{\TODO}[1]{
  \begin{tabular}{l | l}
    \textcolor{red}{TODO} & \parbox{10cm}{\textcolor{red}{{#1}}} \\
  \end{tabular}
}

%can/should? author and title go to the preamble?
\author{Giles Egidijus Nedzinskas}
\title{Design Document for \Projectname{}}
\maketitle{}

\tableofcontents

\chapter{Overview}
\Projectname{} is a project encompassing a game-engine (\Enginename{}) and a game written on it (\Gamename{}).

\Projectname{} begins with the two-pronged passion of one person that for one - thirsts for knowledge and technical prowess in programming and believes that a game-engine is a great and multi-faceted enough challenge to embark on in hopes of becoming a better professional in the software development industry, and two - has a great passion for games and fiction and hopes to use the game as a creative outlet in addition to the creative challenge provided by game engine development.

\Gamename{} is a tile-based, turn-based RPG, inspired by the looks of games made in the 90s and the look and feel of many RPG games made since then. The game is developed alongside the engine \Enginename{} and together with it forms the main and biggest portion of \Projectname{}.

\section{Planned Features}
\subsection{General Gameplay}
\begin{itemize}
  \item Tiles
  \item Turns
  \item Party based (the player can -- and is encouraged to -- controll more than one character at once)
  \item Passage of time
  \item Time of day and the fun associated with it (e.g. you can hide better at night, but can allso be caught
    unawares if not careful)
  \item Long voyages via different means of transportation
  \item Basic crafting (enchanting, enhancing player gear, etc)
\end{itemize}

\subsection{Combat}
\begin{itemize}
  \item area of effect spells/weapons
  \item long range, mele, thrown weapons
  \item squad tactics
  \item basic cover system (e.g. less accuracy if projectile crosses a non-flat item)
\end{itemize}

\subsection{Graphics}
\begin{itemize}
  \item SDL2 (possibly OpenGL later)
  \item Basic tile/item animations (e.g. torches, water tiles)
\end{itemize}

\subsection{Other}
\begin{itemize}
  \item Multi-genre gameplay: the game is not confined to a single setting and can span sci-fi and fantasy realms in the same grand campaign.
\end{itemize}

\section{Possible Features}
\begin{itemize}
  \item Isometric view
  \item Animated characters (e.g. movement, combat)
  \item Implementation in OpenGL. Currently plain SDL2 seems like a good choice, but I do want to learn OpenGL and it is quite more popular in the industry.
\end{itemize}

\section{Features that Will Not be Included}
\begin{itemize}
  \item \textbf{Real-time movement and/or combat.} Real-time is outside the scope of the project as it falls outside the feel that \Gamename{} aims to provide.
 
  Nevertheless the challenge of supporting pusable realtime sounds like an interesting and worth-while challenge and would be something quite interesting to implement in the future, maybe in \Enginename{} II.
 
  \item \textbf{3D.} The engine aims to recreate the look and feel of an era when 2D and pre-rendered 3D dominated.
\end{itemize}

\chapter{Workplan}
The project will most likely take us through a few iterations of prototypes before we acquire enough knowledge and know-how to implement the final framework of \Enginename{}.

\begin{itemize}
  \item \Enginename{} 0.1 (a.k.a Prototype I) should provide us with the possibility to create a few basic inter-connected levels. Thus the functionality we should have:
  \begin{itemize}
    \item a basic level editor
    \item a few tiles with simple graphics
    \item it should be possible to load assets from the filesystem (e.g. textures)
    \item definition of a starting point and a player-controlled character that can walk around
    \item we have to be able to save the level into a file (tile settings and textures, level info, etc)
  \end{itemize}
  \item \Enginename{} 0.2 (a.k.a. Prototype II) should allow us to implement player interaction with the gameworld
  \begin{itemize}
    \item doors, windows
    \item containers
    \item a way to program items into the game(weapons, armors) (this does not touch on in-game crafting)
    \item at least partially implemented combat (we can ommit damage types and
      similar things for now)
  \end{itemize}
\end{itemize}

\chapter{Combat}

\section{Mechanics}
Current (very early) idea of combat. Damage calculated for each damage type
separately using the following logic and equations.

\subsubsection*{Step 1}
Attack and defence scores are calculated (random value up to stat maximum):
\begin{equation*} atk \sim U([0,1 \dots atk_{max}) \end{equation*}
\begin{equation*} def \sim U([0,1 \dots def_{max}) + dodge\end{equation*}
The attack connects \begin{math} if(atk > def) \end{math}. 
\subsubsection*{Step 2}
If the attack is successfull , we calculate actual damage received. First,
incoming damage and armor values are rolled for the current attack.
\begin{equation*} dmg_i \sim U([dmg_{min}, dmg_{min}+1 \dots dmg_{max}]) \end{equation*}
\begin{equation*} arm \sim U([0,1 \dots arm_{max}) \end{equation*}
Then armor is substracted from damage (including resistences or weaknesses
incurred by the armor (if worn)), and we have our final incoming damage to the entity.
The final damage received is still subject to further modification due to entity
resistences or weaknesses to the incoming damage type.
\begin{equation*} dmg = (dmg_i - arm \times resist_{arm}) \div resist_{entity} \end{equation*}

\TODO{The weapon's damage randomization should lean towards midrange (e.g. like
  3d6 values have low percentage to be 3 or 18, but highest percentage to fall
  around 12) and not rely on dice, but on value ranges. That said, maybe we
  should not avoid dice after all, dice values have a nice RPG feel.}

%Something like this (pseudocode):
%\begin{minted}{c++}
%  const int atk = rand(0,attacker.atk);
%  const int def = rand(0,defender.atk);
%  if(def >= atk) { return; }
%
%  const int dmg_i = rand(attacker.weapon.dmg_min, attacker.weapon.dmg_min);
%  const int arm = rand(attacker.weapon.dmg_min, attacker.weapon.dmg_min);
%  int dmg = (dmg_i - arm * defender.armor.resist) * defender.resist;
%  if(dmg < 0) { dmg = 0; }
%\end{minted}

\subparagraph*{Notes}
\begin{itemize}
  \item This logic does not represent catering for damage types, I belive we can
    calculate each damage type separately calling the above logic once per
    damage type.
  \item the armor score is randomized to emulate lucky hits, etc. If we decide to
    suport hit-locations we may want to make the armor score non-random or reduce
    the range coloser to the max value (to still emulate some luck, then again
    luck can be represented with critical strikes).
\end{itemize}

\section{Item Examples}

\subsubsection{weapons}
\rowcolors{1}{milkpink}{rose}
\begin{tabu}{l c c c l}
  %\firsthline
  \rowfont{\bfseries\itshape}
  name               & damage     & cost     & weight & notes \\
  \hline
  axe                & sw+2 cut   & \${}50   & 4      &       \\
  auto pistol 9mm    & 2d6+2 pi   & cost     & weight & [1]   \\
  %\lasthline
\end{tabu}

\subsubsection{armor}
\rowcolors{1}{milkpink}{rose}
\begin{tabu}{l c c c c l}
  %\firsthline
  \rowfont{\bfseries\itshape}
  name               & location     & DR       & cost     & weight & notes   \\
  \hline
  scale armor        & torso, groin & 4        & \${}420  & 35     &         \\
  ballistic suit     & body, limbs  & 12/$4^*$ & cost     & weight & [1,2,3] \\
  %\lasthline
\end{tabu}

\TODO{Make example combat situations with some armor and mele and ranged wepons}

\TODO{Move example item tables to basic equipment section, add a few more items}

\TODO{Make explanations of damage and DR notation}

\end{document}