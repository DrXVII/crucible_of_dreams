\documentclass[a4paper,10pt]{book}
\usepackage[utf8]{inputenc}
\usepackage{mathtools}
\usepackage{minted}
\usepackage{hyperref}
\usepackage[table]{xcolor}

\definecolor{red}{HTML}{FF0000}
\definecolor{globaltablecolor}{HTML}{FF0000}
\definecolor{blue}{HTML}{0000FF}
\definecolor{orange}{HTML}{FF8000}

\begin{document}
% new commands ------------------------------------------------------------------
% - name variables --------------------------------------------------------------
\newcommand{\Projectname}{Project Crucible}
\newcommand{\Enginename}{Crucible}
\newcommand{\Gamename}{Library of Worlds}
% - special formats -------------------------------------------------------------
% TODO - make similar functions for, other admonitions,
% like NOTE, WARNING, INFO, TIP, SUGGESTION etc.
\newcommand{\admonition}[3][white]{
  %#1 - background color %#2 - admonition name #3 - admonition text
  \begin{table*}[h]
  \colorbox{#1}{
  \begin{tabular}{l | l}
    {#2} &
    \begin{minipage}{\linewidth - \evensidemargin - \tabcolsep}
      {#3}
    \end{minipage}
  \end{tabular}
  }
  \end{table*}
}
\newcommand{\TODO}[1]{
  \admonition[red!10]{\textcolor{red}{TODO}}{\textcolor{red}{#1}}
}
\newcommand{\note}[1]{\admonition[blue!10]{NOTE}{#1}}
\newcommand{\think}[1]{\admonition[orange!15]{THNK}{\textcolor{orange}{#1}}}
% - other -----------------------------------------------------------------------
\newcommand{\textbi}[1]{\textbf{\textit{#1}}}

\author{Giles Egidijus Nedzinskas}
\title{Design Document for \Projectname{}}
\maketitle{}

\tableofcontents

\part{Features}

\chapter{Overview}
\Projectname{} is a project encompassing a game-engine (\Enginename{}) and a game written on it (\Gamename{}).

\Projectname{} begins with the two-pronged passion of one person that for one - thirsts for knowledge and technical prowess in programming and believes that a game-engine is a great and multi-faceted enough challenge to embark on in hopes of becoming a better professional in the software development industry, and two - has a great passion for games and fiction and hopes to use the game as a creative outlet in addition to the creative challenge provided by game engine development.

\Gamename{} is a tile-based, turn-based RPG, inspired by the looks of games made in the 90s and the look and feel of many RPG games made since then. The game is developed alongside the engine \Enginename{} and together with it forms the main and biggest portion of \Projectname{}.

\section{Planned Features}
\subsection{General Gameplay}
\begin{itemize}
  \item Tiles
  \item Turns
  \item Party based (the player can -- and is encouraged to -- controll more than one character at once)
  \item Passage of time
  \item Time of day and the fun associated with it (e.g. you can hide better at night, but can allso be caught
    unawares if not careful)
  \item Long voyages via different means of transportation
  \item Basic crafting (enchanting, enhancing player gear, etc)
  \item Lighting levels (e.g. tiles closer to a torch are better lit, caves
    are naturally dark, etc)
  \item Walls between tiles - the walls do not have to occupy an entire tile
\end{itemize}

\subsection{Combat}
\begin{itemize}
  \item magic
  \item area of effect spells/weapons (long duration, instant, hidden(traps -
    could be a trap-spell like a firemine-rune or something more mundane))
  \item long range, mele, thrown weapons
  \item squad tactics
  \item basic cover system (e.g. less accuracy if projectile crosses a non-flat item)
\end{itemize}

\subsection{Graphics}
\begin{itemize}
  \item isometric tiles
  \item SDL2 (possibly OpenGL later)
  \item Basic tile/item animations (e.g. torches, water tiles)
\end{itemize}

\subsection{Other}
\begin{itemize}
  \item Multi-genre gameplay: the game is not confined to a single setting and can span sci-fi and fantasy realms in the same grand campaign.
\end{itemize}

\section{Possible Features}
\begin{itemize}
  \item Isometric view
  \item Animated characters (e.g. movement, combat)
  \item Implementation in OpenGL. Currently plain SDL2 seems like a good choice, but I do want to learn OpenGL and it is quite more popular in the industry.
\end{itemize}

\section{Features that Will Not be Included}
\begin{itemize}
  \item \textbf{Real-time movement and/or combat.} Real-time is outside the scope of the project as it falls outside the feel that \Gamename{} aims to provide.
 
  Nevertheless the challenge of supporting pusable realtime sounds like an interesting and worth-while challenge and would be something quite interesting to implement in the future, maybe in \Enginename{} II.
 
  \item \textbf{3D.} The engine aims to recreate the look and feel of an era when 2D and pre-rendered 3D dominated.
\end{itemize}

\chapter{Workplan}
The project will most likely take us through a few iterations of prototypes before we acquire enough knowledge and know-how to implement the final framework of \Enginename{}.

\begin{itemize}
  \item \Enginename{} 0.1 (a.k.a Prototype I) should provide us with the possibility to create a few basic inter-connected levels. Thus the functionality we should have:
  \begin{itemize}
    \item a basic level editor
    \item a few tiles with simple graphics
    \item it should be possible to load assets from the filesystem (e.g. textures)
    \item definition of a starting point and a player-controlled character that can walk around
    \item we have to be able to save the level into a file (tile settings and textures, level info, etc)
  \end{itemize}
  \item \Enginename{} 0.2 (a.k.a. Prototype II) should allow us to implement player interaction with the gameworld
  \begin{itemize}
    \item doors, windows
    \item containers
    \item a way to program items into the game(weapons, armors) (this does not touch on in-game crafting)
    \item at least partially implemented combat (we can ommit damage types and
      similar things for now)
  \end{itemize}
\end{itemize}

\chapter{Combat}

\section{Mechanics}
Current (very early) idea of combat. Damage calculated for each damage type
separately using the following logic and equations.

\subsubsection*{Step 1}
Attack and defence scores are calculated (random value up to stat maximum):
\begin{equation*} atk \sim U([0,1 \dots atk_{max}) \end{equation*}
\begin{equation*} def \sim U([0,1 \dots def_{max}) + dodge\end{equation*}
The attack connects \begin{math} if(atk > def) \end{math}. 
\subsubsection*{Step 2}
If the attack is successfull , we calculate actual damage received. First,
incoming damage(based on weapon dmg dice) and armor values are rolled for the current attack.
\begin{equation*} dmg_i = dicethrow(\{weapon_{dice}\}) + weapon_{constdmg} \end{equation*}
\begin{equation*} arm \sim U([0,1 \dots arm_{max}) \end{equation*}

\note{what did I want to note again?}

Then armor is substracted from damage (including resistences or weaknesses
incurred by the armor (if worn)), and we have our final incoming damage to the entity.
The final damage received is still subject to further modification due to entity
resistences or weaknesses to the incoming damage type.
\begin{equation*} dmg = (dmg_i - arm \times resist_{arm}) \div resist_{entity} \end{equation*}

\TODO{The weapon's damage randomization should lean towards midrange (e.g. like
  3d6 values have low percentage to be 3 or 18, but highest percentage to fall
  around 12) and not rely on dice, but on value ranges. That said, maybe we
  should not avoid dice after all, dice values have a nice RPG feel.}

%Something like this (pseudocode):
%\begin{minted}{c++}
%  const int atk = rand(0,attacker.atk);
%  const int def = rand(0,defender.atk);
%  if(def >= atk) { return; }
%
%  const int dmg_i = rand(attacker.weapon.dmg_min, attacker.weapon.dmg_min);
%  const int arm = rand(attacker.weapon.dmg_min, attacker.weapon.dmg_min);
%  int dmg = (dmg_i - arm * defender.armor.resist) * defender.resist;
%  if(dmg < 0) { dmg = 0; }
%\end{minted}

\subparagraph*{Notes}
\begin{itemize}
  \item This logic does not represent catering for damage types, I belive we can
    calculate each damage type separately calling the above logic once per
    damage type.
  \item the armor score is randomized to emulate lucky hits, etc. If we decide to
    suport hit-locations we may want to make the armor score non-random or reduce
    the range coloser to the max value (to still emulate some luck, then again
    luck can be represented with critical strikes).
\end{itemize}

\section{Item Examples}
\begin{table*}[h]
  \rowcolors{3}{globaltablecolor!6}{globaltablecolor!12}
  \begin{tabular}{l c c c l}
    %\firsthline
    \textbf{weapons} \\
    \textbi{name}      & \textbi{damage}     & \textbi{cost}     & \textbi{weight} & \textbi{notes} \\
    axe                & sw+2 cut   & \${}50   & 4      &       \\
    auto pistol 9mm    & 2d6+2 pi   & cost     & weight & [1]   \\
    %\lasthline
  \end{tabular}

  \rowcolors{3}{globaltablecolor!6}{globaltablecolor!12}
  \begin{tabular}{l c c c c l}
    %\firsthline
    \textbf{armor} \\
    \textbi{name}      & \textbi{locationq} & \textbi{DR}   & \textbi{cost} & \textbi{weight} & \textbi{notes} \\
    scale armor        & torso, groin & 4        & \${}420  & 35     &         \\
    ballistic suit     & body, limbs  & 12/$4^*$ & cost     & weight & [1,2,3] \\
    %\lasthline
  \end{tabular}
\end{table*}

\TODO{
  \begin{itemize}
    \item Make example combat situations with some armor and mele and ranged wepons
    \item Move example item tables to basic equipment section, add a few more items
    \item Make explanations of damage and DR notation
  \end{itemize}
}

\chapter{Taking Turns}
\TODO{
  Things to ponder
  \begin{itemize}
    \item How will we represent speed, ecpecially in combat (time units in
      combat mode?)
    \item we should probably have a peacefull mode and a combat mode (as we want
      to be able to move more quiclky while not in combat (go to mouse click), etc)
    \item what actions will take TUs (e.g. easy and hard to access inventory
      items, attacks, spell preparation, arming grenades, etc)
  \end{itemize}
}

\section{Combat}
Combat will be squad-style: the player team takes a turn, then the environment
takes a turn. Possibly we will want to support several teams, if -- for example
-- we have several hostile parties with different friend-foe logic and agenda,
like where there are guards that are hostile towards the player's company for
crime, but there are also bandits on the scene that are hostile towards both the
guards and the player's party, and there is allso that sneaking pack of starved
lions who want dinner and nobody noticed them yet.

\section{Peacefull Mode}
\begin{itemize}
\item resting
  \begin{itemize}
  \item fast movement (skip turns untill selected character reaches destination
    (mouseclick)
    \item no need to move each party member, they group around selected party
      member
    \item do we need formations? probably just gonna use line formation for simplicity.
      If I recall correctly D\&D used to make people choose who is where in the
      line without ever mentioning oher formations.
  \end{itemize}
\end{itemize}

\chapter{Game Levels}
\section{Static Pieces}
(e.g.  tiles, walls)
\section{Interactable Pieces}
(e.g. doors, windows, workbenches, drawbridges)
\section{Items, Containers, Furniture}
\section{Decorations}
(e.g. floor decals, etc)
\section{Destructability}
(e.g. burning bridge)
\section{Lighting Levels}
\TODO{This section will be abou lighting levels and how it affects gameplay, the
associated abilities and some possible situations and circumstances of gameplay
interest}

\chapter{Travelling}
\think{Maybe we should simplify travelling a bit? It will not happen all that
  much in the game, at least the vehicles part\dots{} wil it?}
\section{Travel Speed, Transportation and Supplies}
\subsection{Seafairing}
\subsection{Passing Mountains}
\subsection{Travelling Through Forests}
\subsection{Deserts}
\subsection{Flying}
\subsection{Transport animals}
\subsection{Transport vehicles}
\subsubsection{Muscle Powered (e.g. bicycle or animal-drawn)}
\subsubsection{Powered}
Energy cells, solar, other natural source, fuel (e.g. gasolene). There are
different consideration for energy powered vehicles than fueled vehicles.
\subsection{Rest}
\subsection{Pathfinding}
\subsection{Tracking and Being Tracked}
\subsection{Time of Day}
\subsection{Interdimentional Travel}
\section{Difficult Terrain}
\subsection{Mountaineering}
\subsection{Forestry}
\subsection{Swamps}
\subsection{Desert Survival}
\section{Encounters}
\subsection{Area Danger and Peace Level}
\note{there may be a restless area where encounters are likely, but are less
  dangerous for a seasoned advenurer, but there may be peacefull places where
  encounters are rare but may be very dangerous when they do happen}
\subsection{Random Encounters}
\subsection{Preset Encounters}

\chapter{Other Features}

\section{Crafting, Enhancing and Enchanting}
Crafting will not be a large feature in the game, so we group it together with
other topics that change equipment properties.
\note{Maybe crafting will not be in the game at all.}

\part{The Setting}
\TODO{gather the scattered notes here and start from there}
%could be more than one world, probably a chapter for each major world then
\chapter{World}
\section{Magic}
\section{Races}
\section{Items}
\subsection{Weapons}
\subsection{Armor}
\subsection{Arcane}
\chapter{Minor Races}
\chapter{Beasts}
\TODO{The chapter name ``Beasts'', implies less than we want to cover here,
  ``Bestiary'' or ``Other Creatures'' or ``Book of Creatures'' would be
  something more like it}
\section{The Wilds}
\section{The Underworld}
\section{The Void}

\part{The Story}
\TODO{gather the scattered notes here and start from there}

\end{document}